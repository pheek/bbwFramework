%% Philipp G Freimann Juli 2019 für die BBW
%% Phi BBW-Vorlage für Mathematische Dokumente (LaTeX)
%% 2019 - 07 - 11
%% 2023 - 09 - 09  Änderungen für BMP Layout

\documentclass[twoside,12pt,a4paper]{article}%%
\usepackage[paper=a4paper,margin=17mm]{geometry}

%% Zentralisiert

%% Welche Zielgruppe soll ausgedruckt werden?
%% Typischerweise entweder TALS oder GESO (nicht beide). TRAINER ist optional für
%% die Trainer Version.

\ifdefined\isTRANSITIONAL%
\else%
%%
%% Diese "if"s werden benötigt, um die
%% Versionen zu steuern

\newif\ifisTALS
\newif\ifisGESO
\newif\ifisTRAINER



%
\fi%

\input{inputs/bmsUsePackages}
%%%%%%%%%%%%%%%%% L A Y O U T  %%%%%%%%%%%%%%%%%%%%%%%%%%%%
%% 2020-12-27 ph. g. freimann @ bbw.ch
%%

\fancyhf{}%%
\pagestyle{fancy}%%

\renewcommand{\sectionmark}[1]{%%
  \markboth{\thesection{} \ #1}{}%%
}%%

\renewcommand{\subsectionmark}[1]{%%
  \markright{\thesubsection \ #1}%%
}%%

%% Achtung: chaptermark nur im BOOK-Style

\renewcommand{\footrulewidth}{0.4pt}

\parskip4pt
\parindent0pt

%\topmargin-2.0cm
\textheight24.4cm

\renewcommand{\arraystretch}{1}%%

\newenvironment{bbwFillInTabular}{%%
%% BEGIN PART:
\renewcommand{\arraystretch}{2.1}
\begin{tabular}%%
}%% END PART:
{\end{tabular}
\renewcommand{\arraystretch}{1}%%
}%% END Environment bbwFillInTabular


%% left margin reducing:
%% from here: https://stackoverflow.com/questions/1670463/latex-change-margins-of-only-a-few-pages
\newenvironment{changemargin}[4]{%
\begin{list}{}{%
\setlength{\topsep}{0pt}%
\setlength{\leftmargin}{#1}%
\setlength{\rightmargin}{#2}%
\setlength{\topmargin}{#3}%
\setlength{\textheight}{#4}
\setlength{\listparindent}{\parindent}%
\setlength{\itemindent}{\parindent}%
\setlength{\parsep}{\parskip}%
}%
\item[]}{\end{list}}

\input{hyphenAll}

\input{inputs/bmsMakros}
%% Philipp G Freimann Juli 2019 für die BBW
%% Phi BBW-Vorlage für Mathematische Dokumente (LaTeX)
%% 2019 - 07 - 11
%%%%%%%%%%%%%%%%%%%%%%%%%%% M a t h e   M a k r o s %%%%%%%%%%%%%%%%%%%%%%%%%%%%%5

\usetikzlibrary{arrows.meta}


%% Kleine Symbole über anderen. Z. B. "?" über einem
%% Gleichheitszeichen:
%% Use \ueberMini{=}{?} um ein kleines Fragezeichen über ein
%% Gleichheitsszeichen zu schreiben.
\newcommand{\ueberMini}[2]{ \mathrel{\stackrel{\makebox[0pt]{\mbox{\normalfont\tiny
#2}}}{#1}} }

%% Gleichungssystem mit zwei Zeilen und vier Einträgen (je zwei links
%% bzw. rechts).
\def\gleichungZZ#1#2#3#4{%%
  $$\left|
  \begin{array}{rcl}
    {#1} &=& {#2}\\
    {#3} &=& {#4}
    \end{array}\right|$$}%%

\def\gleichungDD#1#2#3#4#5#6{%%
  $$\left|
  \begin{array}{rcl}
    {#1} &=& {#2}\\
    {#3} &=& {#4}\\
    {#5} &=& {#6}\\
    \end{array}\right|$$}%%

\def\gleichungsSystemNummeriert#1#2#3#4{
\begin{center}%%
\begin{tabular}{|ccr|c}
$#1$ & $=$ & $#2$ & \TRAINER{(I)}\\
$#3$ & $=$ & $#4$ & \TRAINER{(II)}
\end{tabular}%%
\end{center}
}%% end gleichungsSystem





%% Entspricht-Symbol
%%\usepackage{accents}
\newcommand{\hatset}[1]{\accentset{\wedge}{#1}}
\newcommand{\entspricht}{\,\,\hatset{=}\,\,}
\newcommand*\mittelwert[1]{\bar{#1}}
\newcommand*\mediantilde[1]{\widetilde{#1}}
%% Das Gradzeichen brauche ich oft:
\newcommand{\degre}{\ensuremath{^\circ}}


%% Eulersche Konstante im Deutschen aufrecht:
 \DeclareMathOperator{\e}{\mathrm{e}}




%% Lösungsmenge für x
\newcommand{\LoesungsMenge}{L}
\newcommand{\lx}{\mathbb{\LoesungsMenge{}}_x}


%vectors:
%%\AtBeginDocument{\renewcommand{\vec}[1]{\overrightarrow{#1}}}

%%% Spaltenvektor
%% \Spvec{2;-1} oder \Spvec[c]{3;c;-4}
%% from here: https://tex.stackexchange.com/questions/2705/typesetting-column-vector
\makeatletter
\newcommand{\Spvek}[2][r]{%
  \gdef\@VORNE{1}
  \left(\hskip-\arraycolsep%
    \begin{array}{#1}\vekSp@lten{#2}\end{array}%
  \hskip-\arraycolsep\right)}

\def\vekSp@lten#1{\xvekSp@lten#1;vekL@stLine;}
\def\vekL@stLine{vekL@stLine}
\def\xvekSp@lten#1;{\def\temp{#1}%
  \ifx\temp\vekL@stLine
  \else
    \ifnum\@VORNE=1\gdef\@VORNE{0}
    \else\@arraycr\fi%
    #1%
    \expandafter\xvekSp@lten
  \fi}
\makeatother




%%
%% Graphiken mit tikz: BBW-Mathe-akros
%%
\tikzset{bbwgrid/.style={help lines,color=bbwMMFarbe!25,thick,step=0.5cm}}

%% einfach die folgende Zeile neu definieren bei kleineren Graphen und
%% scale auf z. B. 0.5 setzen
\tikzset{graphSkalierung/.style={xscale=1,yscale=1}}

%% Koordinatensystem ohne Zahlen
\newcommand{\bbwGridPartLeer}[4]{
 % grid:
 \draw[bbwgrid,graphSkalierung] (#1,#3) grid (#2,#4);

 % axes
 \draw[thick,graphSkalierung] (#1,0) -- (#2,0);
 \draw[thick,graphSkalierung] (0,#3) -- (0,#4);
 \foreach \x in {#1, ..., -1}  \draw[graphSkalierung] (\x cm, 3pt) -- (\x cm, -3pt);%%  node[anchor=north]{$\x$};
 \foreach \x in {1, ..., #2}   \draw[graphSkalierung] (\x cm, 3pt) -- (\x cm, -3pt);%%  node[anchor=north]{$\x$};
 \foreach \y in {#3, ..., -1}  \draw[graphSkalierung] (-3pt, \y cm) -- (3pt, \y cm);%%  node[anchor=east]{$\y\,\,$};
 \foreach \y in {1, ..., #4}   \draw[graphSkalierung] (-3pt, \y cm) -- (3pt, \y cm);%%  node[anchor=east]{$\y\,\,$};
 \draw[->,thick,graphSkalierung] (#2,0) -- ({#2+0.5},0) node[anchor=west]{$x$};
 \draw[->,thick,graphSkalierung] (0,#4) -- (0,{#4+0.5}) node[anchor=south]{$y$};
}

\newcommand{\bbwGridPart}[4]{
 % grid:
 \draw[bbwgrid,graphSkalierung] (#1,#3) grid (#2,#4);

 % axes
 \draw[thick,graphSkalierung] (#1,0) -- (#2,0);
 \draw[thick,graphSkalierung] (0,#3) -- (0,#4);
 \foreach \x in {#1, ..., -1}  \draw[graphSkalierung] (\x cm, 3pt) -- (\x cm, -3pt)  node[anchor=north,graphSkalierung]{\small $\x\,\,\,$};
 \foreach \x in {1, ..., #2}   \draw[graphSkalierung] (\x cm, 3pt) -- (\x cm, -3pt)  node[anchor=north,graphSkalierung]{$\x$};
 \foreach \y in {#3, ..., -1}  \draw[graphSkalierung] (-3pt, \y cm) -- (3pt, \y cm)  node[anchor=east,graphSkalierung]{\small $\y\,\,$};
 \foreach \y in {1, ..., #4}   \draw[graphSkalierung] (-3pt, \y cm) -- (3pt, \y cm)  node[anchor=east,graphSkalierung]{$\y\,\,$};
 \draw[->,thick,graphSkalierung] (#2,0) -- ({#2+0.5},0) node[anchor=west]{$x$};
 \draw[->,thick,graphSkalierung] (0,#4) -- (0,{#4+0.5}) node[anchor=south]{$y$};
 }


%% A function within a Grid (without painting the grid)
%% #1: funciton eg 2*\x  (x has to be backquoted)
%% #2: Domain eg. -1:2.5
%% #3: colour eg red
\newcommand{\bbwFuncC}[3]{\draw[domain=#2,smooth,samples=200,variable=\x,#3] plot ({\x},{#1});
}
%% A function within a Grid (without painting the grid)
\newcommand{\bbwFunc}[2]{\bbwFuncC{#1}{#2}{blue}}

%% Declare a function-plot
%% xmin,xmax,ymin,ymax,fct,domain(x-min, x-max)
%% example: \bbwFunction{-4}{3}{-2}{5}{-\x*\x- \x + 4.5}{-3:2}
\newcommand{\bbwFunction}[6]{\begin{tikzpicture}
\bbwGridPart{#1}{#2}{#3}{#4}
 \bbwFunc{#5}{#6}
%% \draw[domain=#6,smooth,samples=200,variable=\x,blue] plot ({\x},{#5});
\end{tikzpicture}}
%% a whole graph having a coordinate-system #1-#4 and any tizpicture content #5:
\newcommand{\bbwGraph}[5]{\begin{tikzpicture}\bbwGridPart{#1}{#2}{#3}{#4}#5\end{tikzpicture}}
\newcommand{\bbwGraphLeer}[5]{\begin{tikzpicture}\bbwGridPartLeer{#1}{#2}{#3}{#4}#5\end{tikzpicture}}

%% Dots and lines:
%% Dot example: \bbwDot{-1,2}{red}{east}{A}
\newcommand{\bbwDot}[4]{\filldraw[color=#2!60, fill=#2!5, thick](#1) circle(0.05) node[anchor=#3]{$#4$};}

%% Line example: \bbwLine{-1,0}{2,3}{red}
\newcommand{\bbwLine}[3]{\draw[thick,color=#3] (#1)--(#2);}

\newcommand{\bbwArrow}[3]{\draw[thick,color=#3,->] (#1)--(#2);}

%% Strecke mit zwei Endpunkten
%% #1, #2: Koordinaten der Endpunkte
%% #3: Ort der Beschriftung
%% #4: Beschtiftung (z. B. "a)")
%% #5: farbe der Strecke und der Beschriftung
\newcommand{\bbwStrecke}[5]{%%
\bbwLine{#1}{#2}{#5}%
\bbwDot{#1}{#5}{west}{}%
\bbwDot{#2}{#5}{west}{}%
\draw (#3) node{{\color{#5}#4}};
}%%



%% Draw a single letter or small text
% #1: Position eg  4,4
% #2: letter eg f or blah
% #3: colour
\newcommand{\bbwLetter}[3]{\draw[color=#3](#1) node{$#2$};}

%%% ABC-Formel
%% usage \abcd{<a>}{<b>}{<c>}
%% example usage: \abcd{b}{5}{\sqrt{4}}
\newcommand{\abcd}[3]{$\frac{-(#2)\pm\sqrt{(#2)^2 - 4\cdot{}(#1)\cdot{}(#3)}}{2\cdot{}(#1)}$}


%% coordSysBBWFlex
%% Flexibles Koordinatensystem mit Pfeilen und Pfeilbeschriftung, aber
%% noch ohne "ticks".
%% #1   : Rastergröße
%% #2-#5: Größe des Rasters in cm
%% #6   : Beschriftung in x-Richtung (in y-Richtung ist es immer y
%% #7   : Zu zeichnende Funktion
%% #8   : Ticks oder was sonst noch komplexeres in die Grafik muss
\newcommand{\bbwFunctionColour}{blue}
\newcommand{\coordSysBBWFlex}[8]{
\begin{tikzpicture}
\draw[step = #1,  thin, cyan!20] (#2, #4) grid (#3, #5);
\draw[thick, ->] (#2,0) -- (#3,0) node[anchor = west] {$#6$};
\draw[thick, ->] (0,#4) -- (0,#5) node[anchor = south] {$y$};
\draw[domain=#2:#3,smooth,samples=200,variable=\x,\bbwFunctionColour] plot ({\x},{#7});
#8;
\end{tikzpicture}
\renewcommand{\bbwFunctionColour}{blue}
}%% end coordSysBBW

\newcommand{\einheitskreis}{
\definecolor{qqwuqq}{rgb}{0,0.79,0}
\definecolor{qqqqff}{rgb}{0,0,1}
\definecolor{qqzzzz}{rgb}{0,0.6,0.6}
\definecolor{ffwwqq}{rgb}{1,0.4,0}
\definecolor{qqccww}{rgb}{0,0.8,0.4}
\definecolor{uququq}{rgb}{0.25,0.25,0.25}
\definecolor{xdxdff}{rgb}{0.49,0.49,1}
\begin{tikzpicture}[line cap=round,line join=round,>=triangle 45,x=2.5cm,y=2.5cm]
\draw[->,color=black] (-1.5,0) -- (1.5,0);
\foreach \x in {-1.5,-1,-0.5,0.5,1}
\draw[shift={(\x,0)},color=black] (0pt,2pt) -- (0pt,-2pt) node[below] {\footnotesize $\x$};
\draw[->,color=black] (0,-1.5) -- (0,1.5);
\foreach \y in {-1.5,-1,-0.5,0.5,1}
\draw[shift={(0,\y)},color=black] (2pt,0pt) -- (-2pt,0pt) node[left] {\footnotesize $\y$};
\draw[color=black] (0pt,-10pt) node[right] {\footnotesize $0$};
\clip(-1.5,-1.5) rectangle (1.5,1.5);
\draw [shift={(0,0)},color=qqwuqq,fill=qqwuqq,fill opacity=0.1] (0,0) -- (0:0.28) arc (0:46.14:0.28) -- cycle;
\draw(0,0) circle (2.5cm);
\draw [domain=-1.5:1.5] plot(\x,{(-0-0.72*\x)/-0.69});
\draw (0.69,-1.5) -- (0.69,1.5);
\draw (1,-1.5) -- (1,1.5);
\draw (1.12,0.6) node[anchor=north west,color=qqccww] {t};
\draw (0.50,0.42) node[anchor=north west,color=ffwwqq] {s};
\draw (0.30,0.21) node[anchor=north west,color=qqzzzz] {c};
\draw [line width=1.2pt,color=qqccww] (1,1.04)-- (1,0);
\draw [line width=1.2pt,color=ffwwqq] (0.69,0.72)-- (0.69,0);
\draw [line width=1.6pt,color=qqzzzz] (0.69,0)-- (0,0);
\begin{scriptsize}
\fill [color=xdxdff] (0.69,0.72) circle (1.5pt);
\draw[color=xdxdff] (0.49,0.71) node {$P$};
\fill [color=uququq] (0.69,0) circle (1.5pt);
\draw[color=uququq] (0.8,0.12) node {$Xp$};
%%\fill [color=uququq] (1,1.04) circle (1.5pt);
%%\draw[color=uququq] (1.22,0.95) node {$Yp$};
\fill [color=uququq] (0.69,0.72) circle (1.5pt);
\draw[color=uququq] (0.82,0.7) node {$Yp$};
\fill [color=qqqqff] (0,0) circle (1.5pt);
\draw(0.20,0.09) node {$\varphi$};
\end{scriptsize}
\end{tikzpicture}\par
}%% END Command \einheitskreis


\newcommand{\einheitskreisT}{%% Tangens
\definecolor{qqwuqq}{rgb}{0,0.79,0}
\definecolor{qqqqff}{rgb}{0,0,1}
\definecolor{qqzzzz}{rgb}{0,0.6,0.6}
\definecolor{ffwwqq}{rgb}{1,0.4,0}
\definecolor{qqccww}{rgb}{0,0.8,0.4}
\definecolor{uququq}{rgb}{0.25,0.25,0.25}
\definecolor{xdxdff}{rgb}{0.49,0.49,1}
\begin{tikzpicture}[line cap=round,line join=round,>=triangle 45,x=2.5cm,y=2.5cm]
\draw[->,color=black] (-1.5,0) -- (1.5,0);
\foreach \x in {-1.5,-1,-0.5,0.5,1}
\draw[shift={(\x,0)},color=black] (0pt,2pt) -- (0pt,-2pt) node[below] {\footnotesize $\x$};
\draw[->,color=black] (0,-1.5) -- (0,1.5);
\foreach \y in {-1.5,-1,-0.5,0.5,1}
\draw[shift={(0,\y)},color=black] (2pt,0pt) -- (-2pt,0pt) node[left] {\footnotesize $\y$};
\draw[color=black] (0pt,-10pt) node[right] {\footnotesize $0$};
\clip(-1.5,-1.5) rectangle (1.5,1.5);
\draw [shift={(0,0)},color=qqwuqq,fill=qqwuqq,fill opacity=0.1] (0,0) -- (0:0.28) arc (0:46.14:0.28) -- cycle;
\draw(0,0) circle (2.5cm);
\draw [domain=-1.5:1.5] plot(\x,{(-0-0.72*\x)/-0.69});
\draw (0.69,-1.5) -- (0.69,1.5);
\draw (1,-1.5) -- (1,1.5);
\draw (1.12,0.6) node[anchor=north west,color=qqccww] {t};
\draw (0.50,0.42) node[anchor=north west,color=ffwwqq] {s};
\draw (0.30,0.21) node[anchor=north west,color=qqzzzz] {c};
\draw [line width=1.2pt,color=qqccww] (1,1.04)-- (1,0);
\draw [line width=1.2pt,color=ffwwqq] (0.69,0.72)-- (0.69,0);
\draw [line width=1.6pt,color=qqzzzz] (0.69,0)-- (0,0);
\begin{scriptsize}
\fill [color=xdxdff] (0.69,0.72) circle (1.5pt);
\draw[color=xdxdff] (0.49,0.71) node {$P$};
\fill [color=uququq] (0.69,0) circle (1.5pt);
\draw[color=uququq] (0.8,0.12) node {$Xp$};
%%\fill [color=uququq] (1,1.04) circle (1.5pt);
%%\draw[color=uququq] (1.22,0.95) node {$Yp$};
\fill [color=uququq] (0.69,0.72) circle (1.5pt);
\draw[color=uququq] (0.82,0.7) node {$Yp$};
\fill [color=qqqqff] (0,0) circle (1.5pt);

%% Tangens
\draw[color=uququq] (1.27,1) node {$T=(1|t)$};
\fill [color=qqqqff] (1,1.05) circle (1.5pt);


\draw(0.20,0.09) node {$\varphi$};
\end{scriptsize}
\end{tikzpicture}\par
}%% END Command \einheitskreisT


\newcommand{\miniEinheitskreis}{
\tikzset{graphSkalierung/.style={xscale=2,yscale=2}}
\bbwGraph{-1.5}{1.5}{-1.5}{1.5}{
\bbwFuncC{ sqrt(4-\x*\x)}{-2:2}{lightgray}
\bbwFuncC{-sqrt(4-\x*\x)}{-2:2}{lightgray}
}
\tikzset{graphSkalierung/.style={xscale=1,yscale=1}}
}%% end newcommand miniEinheitskreis



\input{inputs/matheMakrosTrigo}

\usepackage{inputs/bms}


%% PBM Prüfung: Kanton: Überschreibe BMS-Winterthur
\renewcommand{\topRightHeaderPruefung}{%%
  \begin{tabular}{cl}

    \makebox{\raisebox{-3.5mm}{\includegraphics[width=5mm]{logos/ZueriWappenCyan.pdf}}
      \hspace{-2mm}}
    &
    \makecell[l]{%%
      \vspace{-1.5mm}{\scriptsize{\cdciFont{Bildungsdirektion}}}\\
      \vspace{-3mm}\scriptsize{\thepage/24}}%%
  \end{tabular}%%
  \vspace{1mm}
}%% end kommand \topRightHeaderPruefung

\renewcommand{\headrulewidth}{0pt}
\renewcommand{\footrulewidth}{0pt}

%%\setlength{\topmargin}{-5mm}
%%\setlength{\leftmargin}{5mm}
%%\setlength{\headheight}{mm}

\usepackage{bmsLayoutPruefungBMPSty}

%%%%%%%%%%%%%%%  H E A D E R   &   F O O T E R %%%%%%%%%%%%%%%%%%%%

\renewcommand{\frageTitelZeile}{%%
\vspace{1mm}\textbf{Aufgabe\, \arabic{frageCounter}\stepcounter{frageCounter}%%
  \hspace*{\fill}\arabic{tmpCounter} Punkte}

\vspace{2mm}%%
}%% end newcommaand frageTitelZeile

\fancyhf[HR]{\topRightHeaderPruefung} 

%\fancyhf[HL]{\makebox{\includegraphics[width=36mm]{logos/bbwBreit.pdf}}}
%\noTRAINER{\fancyhf[HC]{\pruefungsNummer{}. Prüfung: \pruefungsThema{} - \klasse}}
%\TRAINER{\fancyhf[HC]{\colorbox{red}{\pruefungsNummer{}. Prüfung: \pruefungsThema{} - \klasse}}}
%\fancyhf[FR]{\tiny{fp @ bbw (Druck: \today{})}}


%% dies wäre der korrekte font (sofern installiert), doch SEHR fett???
%\usepackage{fontspec}
%\setmainfont[Mapping=tex-text-ms]{4213-font}

\newcommand{\ausrichtung}{AUSRICHTUNG überschreiben}
\newcommand{\pruefungsIDAufgabe}{prüfungsIDAufgabe überschreiben}

\fancyhf[FL]{\tiny{Diese Prüfungsaufgaben dürfen erst nach Freigabe durch die Kommission Berufsmaturität (KBM) im Unterricht verwendet werden.\\
Eine kommerzielle Verwendung bedarf der Bewilligung der KBM des Kantons Zürich.}}
\fancyhf[FR]{\tiny{\pruefungsIDAufgabe{}\\\ausrichtung{}}}
