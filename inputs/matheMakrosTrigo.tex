%% Trigo Koordinatensysteme



%% Trigonometrische Koordinatensysteme
%% Alle heißen "trigsysS" wobei da S einer der folgenden Sub-Systeme
%% bezeichnet:
%%  A  phi von  0 ... 360
%%     y   von -3 ...   3
%%
%%  B  phi von  0 ... 360
%%     y   von -1 ...   1
%%
%%  C  phi von  -270 ... 450
%%     y   von    -2 ...   2
%%
%%  D  phi von  -270 ... 450
%%     y   von    -1 ...   1
%%


%% Koordinatensystem von 0 - 360 Grad (y -Ricthung -1 bis 1
%% Die Funktion kann mit dem 1. Parameter eingegeben werden

\newcommand{\trigsysAFct}[1]{
\coordSysBBWFlex{0.5cm}{-1}{13}{-4}{4}{\varphi}{#1}{
  \foreach \x [evaluate=\x as \degree using int(\x*30)] in {1,...,12}{ 
    \draw (\x cm, 1pt) -- (\x cm, -1pt) node[anchor = north] {$\degree^\circ$};
  }
  \foreach \y in {-3,-2,-1,1,2,3}{
   \draw (1pt, \y cm) -- (-1pt, \y cm) node[anchor = east] {$\y$};
  }
}
}%% end trigsysC

%% Leeres Koordinatensystem (fct = 0)
\newcommand{\trigsysA}{\trigsysAFct{0}}


%% Koordinatensystem von -270 bis 450 Grad. In y-Richtung von -2 bis 2
%% Funktion wird mit #1-Parameter angegeben
\newcommand{\trigsysBFct}[1]{
\coordSysBBWFlex{0.5cm}{-1}{13}{-4}{4}{\varphi}{#1}{
  \foreach \x [evaluate=\x as \degree using int(\x*30)] in {1,...,12}{ 
    \draw (\x cm, 1pt) -- (\x cm, -1pt) node[anchor = north] {$\degree^\circ$};
  }
  \foreach \y in {-1,1}{
   \draw (1pt, \y *3cm) -- (-1pt, \y *3cm) node[anchor = east] {$\y$};
  }
}
}%% end trigsysC

%% Leeres B-System
\newcommand{\trigsysB}{\trigsysBFct{0}}


%% Wie B-SYstem, jedoch in y-Richtung von -1 bis +1
\newcommand{\trigsysCFct}[1]{
\coordSysBBWFlex{0.2cm}{-6}{10}{-2.5}{2.5}{\varphi}{#1}{
  \foreach \x [evaluate=\x as \degree using int(\x*90)] in {-3,-2,-1,1,2,3,4,5}{ 
   \draw (\x *18mm, 1pt) -- (\x * 18mm, -1pt) node[anchor = north] {$\degree^\circ$};
  }
   
  \foreach \y in {-2,-1,1,2}{
    \draw (1pt, \y cm) -- (-1pt, \y cm) node[anchor = east] {$\y$};
  }
}
}%% end trigsysC

\newcommand{\trigsysC}{\trigsysCFct{0}}


\newcommand{\trigsysDFct}[1]{
\coordSysBBWFlex{0.2cm}{-6}{10}{-2.5}{2.5}{\varphi}{#1}{
 \foreach \x [evaluate=\x as \degree using int(\x*90)] in {-3,-2,-1,1,2,3,4,5}{ 
   \draw (\x *18mm, 1pt) -- (\x * 18mm, -1pt) node[anchor = north] {$\degree^\circ$};
  }   
  \foreach \y in {-1,1}
   \draw (1pt, \y *2cm) -- (-1pt, \y *2cm) node[anchor = east] {$\y$};
  }
} %% end command: trig sys D cos()


\newcommand{\trigsysDcos}{\trigsysDFct{2*cos(\x*50)}}
\newcommand{\trigsysDsin}{\trigsysDFct{2*sin(\x*50)}}
\newcommand{\trigsysD}{\trigsysDFct{0}}
