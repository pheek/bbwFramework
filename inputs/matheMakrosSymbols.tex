%% Das "entspricht" - Symbol:
\newcommand{\hatset}[1]{\accentset{\wedge}{#1}}
\newcommand{\entspricht}{\,\,\hatset{=}\,\,}
\newcommand*\mittelwert[1]{\bar{#1}}
\newcommand*\mediantilde[1]{\widetilde{#1}}
%% Das Gradzeichen brauche ich oft:
\newcommand{\degre}{\ensuremath{^\circ}}


%% Eulersche Konstante im Deutschen aufrecht:
 \DeclareMathOperator{\e}{\mathrm{e}}

%% Lösungsmenge für x
\newcommand{\LoesungsMenge}{\mathbb{L}}
\newcommand{\lx}{\LoesungsMenge{}_x}

%% Definitionsmenge
\newcommand{\DefinitionsMenge}{\mathbb{D}}

%% Wertebereich
\newcommand{\Wertebereich}{\mathbb{W}}

%% Wertebereich
\newcommand{\Grundmenge}{\mathbb{G}}

%vectors:
%%\AtBeginDocument{\renewcommand{\vec}[1]{\overrightarrow{#1}}}

%%% Spaltenvektor
%% \Spvec{2;-1} oder \Spvec[c]{3;c;-4}
%% from here: https://tex.stackexchange.com/questions/2705/typesetting-column-vector
\makeatletter
\newcommand{\Spvek}[2][r]{%
  \gdef\@VORNE{1}
  \left(\hskip-\arraycolsep%
    \begin{array}{#1}\vekSp@lten{#2}\end{array}%
  \hskip-\arraycolsep\right)}

\def\vekSp@lten#1{\xvekSp@lten#1;vekL@stLine;}
\def\vekL@stLine{vekL@stLine}
\def\xvekSp@lten#1;{\def\temp{#1}%
  \ifx\temp\vekL@stLine
  \else
    \ifnum\@VORNE=1\gdef\@VORNE{0}
    \else\@arraycr\fi%
    #1%
    \expandafter\xvekSp@lten
  \fi}
\makeatother

%% entspricht
\newcommand{\estimates}{\overset{\scriptscriptstyle\wedge}{=}}
