
%% Wahr/Falsch Boxen
%% Aufruf \wahrbox{wahr}, falls das Resultat wahr ist
%% Aufruf \wahrbox{falsch}, falls ``falsch'' angegeben werden muss.
\newcommand{\wahrbox}[1]{($\Box$ wahr; $\Box$ falsch)\TRAINER{#1}}%%

%% Besser:
%% Aufruf \bbwCheckBox{true}{«hier ist die frage»} falls wahr
%% Aufruf \bbwCheckBox{false}{«hier ist die frage»} falls falsch
\newcommand{\bbwCheckBox}[2]{\ifstrequal{#1}{true}{\TRAINER{\makebox[0pt][l]{$\square$}{\raisebox{0.1\height}{$\times$}}}\noTRAINER{$\Box$}}{$\Box$} #2}

%% Muss etwas schmaler sein, weil jede Frage einen Rahmen darum herum hat.
\renewcommand{\mmPapier}[1]{\mmPapierZwei{#1}{18}}

%% oder einfach \Box falls nicht anzukreuzen und für anzukreuzen:
%% \BoxT
%% Beispiel Kreuzen Sie ja \BoxT{} an und nicht nein \Box{}
\newcommand{\BoxT}{\TRAINER{\color{green} x}\noTRAINER{\Box}}
